\section{Auszeichnung und Prämien}

In jeder Wertung erhalten die drei Erstplatzierten Medaillen. Bis Platz sechs werden Urkunden verliehen. Urkunden aller Plätze sind auf unserer Webseite und in der SwimResults App verfügbar.

Die punktbeste Leistung bei den Frauen und Männern wird mit hochwertiger Schwimmbekleidung und dem Pokal des Bürgermeisters der Bergstadt Marienberg ausgezeichnet. Bei mindestens 900 DSV-Punkten wird eine zusätzliche Geldprämie in Höhe von jeweils 150 Euro, sofern mindestens 2.000 Meldungen von Gastvereinen vorliegen, ausgeschüttet.

Neue Hallenrekorde werden mit Geldprämien von 25 Euro, bei über 900 DSV-Punkten mit 50 Euro honoriert.

Die punktbeste Leistung wird in jeder ausgeschriebenen Wertungsklasse mit einem Pokal sowie einem Badeanzug beziehungsweise einer Badehose geehrt. Bei Punktgleichheit wird die zweitbeste Leistung der Sportler berücksichtigt.
 
In allen Finals werden mind. folgende Geldprämien gezahlt:
1. 120 Euro 	2. 80 Euro 	3. 50 Euro	4. 20 Euro

In den Juniorenfinals (2009 weiblich sowie 2008 männlich und jünger) werden ab 2000 auswärtigen Meldungen Prämien vergeben:
1. 30 Euro	2. 20 Euro	3. 10 Euro	4. 5 Euro

Die Siegerstaffeln erhalten echt erzgebirgische, ewige Wanderpokale.

In allen ausgeschriebenen Wertungen werden folgende Punkte vergeben: 1. 12; 2. 8; 3. 5; 4. 3; 5. 2; 6. 1. Es erhalten in jeder Wertung die beiden besten Aktiven pro Verein Punkte. Die Finals sowie Staffelwertungen werden doppelt bepunktet. Juniorenfinals werden einfach bepunktet. Die punktbeste Mannschaft der gesamten Veranstaltung erhält den Erzgebirgsschwimmcup, die punktbeste Mannschaft des ersten Tages den Junior-Erzgebirgsschwimmcup, beide ewige Wanderpokale. 

Zur 28ten Auflage des IESC werden für die beste Mannschaft 150 Euro, die zweitbeste Mannschaft 100 Euro und das drittplatzierte Team 50 Euro ausgezahlt. Es zählt der Punktestand nach Abschluss aller Wettbewerbe.

Die Siegerehrungen sind Wettkampfbestandteil. Bei Nichtteilnahme erlischt der Anspruch auf alle Auszeichnungen.
