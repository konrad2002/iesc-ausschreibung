\section{Meldungen}

Alle teilnehmenden Vereine sind verpflichtet im DSV-Standard per E-Mail zu melden. Bitte der Meldung einen Kontrollausdruck beifügen. Meldungen ohne Meldezeit oder falschen Angaben werden zurückgewiesen. Die Meldung muss laut Wettkampfbestimmungen des DSV die Erklärung der Sportgesundheit der Aktiven enthalten.

% TODO: unklar, ob pro Wettkampf
Der Veranstalter behält sich vor, Meldungen zurückzuweisen, die Anfangszeiten zu verändern, falls dies die Meldezahl notwendig macht. Über die 200 m Strecken werden jeweils drei Läufe pro Wettkampf zugelassen - über 200 m Lagen pro Wettkampf maximal fünf Läufe. Es starten die Aktiven mit den schnellsten Meldezeiten. Der Ausrichter behält sich vier Starts pro Strecke für seine Aktiven vor. Es zählen Bestleistungen ab \athleteBestDate. Für die 200 m Strecken gleicht der Ausrichter die Meldezeit mit der Bestenliste des DSV ab. Für die 100 m Strecken gelten die Richtzeiten.

\newpage

\textbf{Meldungen müssen bis Dienstag, den \registrationDeadline\ vorliegen.} Nach- und Ummeldungen sind nicht möglich. Um einen schnellen Wettkampfablauf zu gewährleisten, bitten wir, \textbf{Abmeldungen erkrankter Sportler bis Donnerstag, den \deregistrationDeadline}, uns ausschließlich per Mail unbedingt mitzuteilen.

Alle teilnehmenden Vereine erhalten umgehend nach Meldeeingang eine Meldebestätigung. Die Information, ob evtl. Meldungen zurückgewiesen oder die Anfangszeiten verändert werden, erfolgt ggf. bis Sonntag, den \entryConfirmationDate. 

Die Laufeinteilung erfolgt in den Wertungsklassen nach Meldezeiten. Über die 200 m Strecken erfolgt die Laufeinteilung ausschließlich nach Meldezeiten.

Jeder teilnehmende Verein ab 5 Aktiven beziehungsweise 20 Starts ist zur Stellung von einem Kampf- oder Schiedsrichter zu eigenen Lasten verpflichtet. Ab 21 Aktiven beziehungsweise 100 Meldungen möchten die teilnehmenden Vereine bitte zwei Kampfrichter stellen. Die Meldung hat bis Meldeschluss mit Qualifizierungsstufe sowie Einsatzwunsch namentlich zu erfolgen. Alle Kampf- und Schiedsrichter werden verpflegt und erhalten ein Veranstaltungs-T-Shirt.

Ein Meldeergebnis und Protokoll in Papierform werden nur angefertigt, wenn dies bei der Meldung ausdrücklich gewünscht wird. Beides wird auf der Website \textbf{erzgebirgsschwimmcup.de} und in der \textbf{SwimResults App} zur Verfügung gestellt.

Für jede Einzelmeldung (50/100 m) von Vereinen mit mind. 100 Starts und Meldeeingang bis \earlyRegistrationDate\ werden \entryEarly\ Euro berechnet. Für alle anderen Meldungen (50/100 m) werden \entrySingle\ Euro berechnet. Für 200 m Meldungen und je Staffelmeldung werden \entryRelay\ Euro berechnet. Falls Meldungen nicht im gültigen DSV-Standard erfolgen, werden \entryFormatPunishment\ Euro, für nicht gestellte Kampfrichter je \entryRefreePunishment\ Euro pro Meldung berechnet.

{\textbf{
\begin{tabular}{rl}
    Meldeadresse:   & Alexander Steiner\\
			        & Hammergasse 24\\
			        & 09526 Olbernhau\\
			        & Deutschland - Germany\\
	Telefon:        & +49 37 36 07 51 77\\
    Handy           & +49 16 26 80 21 60\\
    E-Mail:         & alex@schwimmteamerzgebirge.de\\
    Web:            & schwimmteamerzgebirge.de
\end{tabular}
}}

Das Meldegeld ist auf das Konto des ST Erzgebirge bei der Volksbank Chemnitz zu überweisen.\\
\textbf{IBAN \bankIBAN\ / BIC \bankBIC}