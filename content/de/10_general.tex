\section{Allgemeine Bestimmungen}

\flagtopright{gm-flag.png}

Die Wettkämpfe werden nach den Wettkampfbestimmungen (WB), der Rechtsordnung (RO), der Wettkampflizenzordnung (WLO) und der Anti-Doping-Ordnung (ADO) des Deutschen Schwimm-Verbandes e.V. (DSV) durchgeführt. Ausrichter und Veranstalter ist das Schwimmteam Erzgebirge. Alle gemeldeten Aktiven, Trainer und Betreuer stimmen der Veröffentlichung von Fotos/Videos zu.

Für Behinderte mit entsprechendem Klassifizierungsnachweis sind zusätzlich die Wettkampfbestimmungen des Deutschen Behinderten-Sportverbandes (DBS) anzuwenden.

Teilnahmeberechtigt sind die Mitglieder von Vereinen/ Startgemeinschaften, die einem dem DSV angeschlossenen Schwimmverband angehören und im Besitz der Verbandsrechte sind, sowie alle Verbände und Vereine des Auslandes, soweit sie über ihre nationalen Verbände Mitglied der FINA sind und das Startrecht haben. Aktive 2015 und jünger dürfen pro Tag inkl.
Staffel maximal sechs Starts absolvieren.
Es kommt die Ein-Start-Regel zur Anwendung.

Für verloren gegangene und beschädigte Gegenstände wird keine Haftung vom Veranstalter, Ausrichter und dem Betreiber der Schwimmstätte übernommen. Die Badeordnung der Einrichtung ist bindend. Bei der Meldung ist die Datenschutzerklärung abzugeben.

Der Internationale Erzgebirgsschwimmcup wurde beim Deutschen (DSV) und Sächsischen (SSV) Schwimmverband angezeigt. 
