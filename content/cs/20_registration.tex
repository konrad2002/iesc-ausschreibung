\section{Přihlášky}

Účastnické oddíly jsou povinny přihlásit se prostřednictvím emailu nebo pomocí elektronického nosiče ve formátu LENEX. K přihlášce přiložte prosím kontrolní výtisk přihlášky. Přihlášky bez uvedení Přihlašovacího času nebo s chybnými údaji budou odmítnuty. Přihláška musí podle Soutěžních pravidel DSV/LENEX obsahovat prohlášení o zdravotním stavu závodníků.

Pořadatel si vyhrazuje právo odmítnout přihlášku, měnit startovní časy, příp. závody jednotlivých ročníků přeložit na jiný den závodu, jestliže to bude vyžadovat počet přihlášených účastníků.  Pro tratě nad 200 m jsou povoleny vždy 3 účasti, pro tratě polohového závodu nad 200 m maximálně 5 účastí v závodě. Startují závodníci s nejlepšími nahlášenými časy. Pořadatel si vyhrazuje 4 starty na každou trať pro své účastníky. Platí nejlepší osobní výkony závodníků od 01.01.2024. Nahlášený čas na 200 m se bude porovnávat se seznamem nejlepších výkonů DSV. Limitní časy platí pro 100 m.

Přihlášky musí být doručeny nejpozději do úterý 02.12.2025. Dodatečné přihlášky nebo změny nejsou možné. V zájmu rychlého průběhu závodu prosíme, abyste nás výhradně prostřednictvím emailu o příp. odhlášení závodníků z důvodu nemoci bezpodmínečně informovali nejpozději do středy 11.12.2025 20:00.

Všechny přihlášené oddíly obdrží informaci o potvrzení přihlášky. Informaci o odmítnutí přihlášky a o změnách začátků závodů obdržíte do neděle 07.12.2025.

Rozvrh závodů bude stanoven ve všech kategoriích podle nahlášených časů závodníků. Rozvrh závodů na tratích přes 200 m bude proveden výlučně na základě nahlášených časů.

Každý oddíl s 5 a více závodníky (resp. 20 starty) je povinen zajistit jednoho rozhodčího. Od 21 závodníků resp. 100 nahlášených startů prosíme o zajištění dvou rozhodčích. Přihláška rozhodčího musí být doručena rovněž do
termínu uzávěrky přihlášek, s vyznačením kvalifikačního stupně a s informací o tom, jakou funkci by chtěl rozhodčí vykonávat. Všichni rozhodčí zde obdrží stravu a pořadatelské tričko.

Potvrzení o přihlášení v tištěné formě poskytneme jen na přání. Oba soubory naleznete na webové stránce erzgebirgsschwimmcup.de a v aplikaci SwimResults.

Za každý jednotlivý vstup (50/100 m) bude účtováno 6 euro pro kluby se 100 a více starty a registracemi přijatými do 31. října 2025. Za všechny ostatní vstupy (50/100 m) bude účtováno 7 euro. Za vstupy na 200 m a štafetové vstupy se platí 10 euro. Pokud nebudou přihlášky provedeny v souladu s platným standardem

DSV, bude účtován příplatek 0,50 euro za přihlášku a příplatek 2 euro za neposkytnuté rozhodčí.

Účastnický poplatek je třeba převést na konto ST Erzgebirge při Volksbank
Chemnitz (IBAN DE23870962140370002800/BIC GENODEF1CH1).

{\textbf{
\begin{tabular}{rl}
    Meldeadresse:   & Alexander Steiner\\
			        & Hammergasse 24\\
			        & 09526 Olbernhau\\
			        & Deutschland - Germany\\
	Telefon:        & +49 37 36 07 51 77\\
    Handy           & +49 16 26 80 21 60\\
    E-Mail:         & alex@schwimmteamerzgebirge.de\\
    Web:            & schwimmteamerzgebirge.de
\end{tabular}
}}
